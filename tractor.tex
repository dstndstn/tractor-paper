\documentclass[linenumbers]{aastex631}

\shorttitle{The Tractor}
\shortauthors{Lang and Hogg}

\newcommand{\thetractor}{\emph{The Tractor}}

\begin{document}

\title{\emph{The Tractor}: forward modeling of astronomical images}

\author[0000-0002-1172-0754]{Dustin A. Lang}
\affiliation{Perimeter Institute for Theoretical Physics, 31 Caroline Street N, Waterloo, ON N25 2YL, Canada}

\author[0000-0003-2866-9403]{David W. Hogg}
\affiliation{Center for Computational Astrophysics, Flatiron Institute, 162 Fifth Avenue, New York, NY 10010, USA}
\affiliation{Max-Planck-Institut f\"ur Astronomie, K\"onigstuhl 17, D-69117 Heidelberg, Germany}
\affiliation{Center for Cosmology and Particle Physics, Department of Physics, New York University, 726 Broadway, New York, NY 10003, USA}

\begin{abstract}
  We present \thetractor, a software package for performing forward or
  \emph{generative} modeling in astronomical images.  Given a set of
  astronomical sources and a model of the properties of an
  astronomical image, \thetractor\ generates a ``model image'': a
  prediction of what would be observed in the image.  In the common
  case where a noise model for the image is available (for example,
  typical ground-based background-limited optical images), it is
  simple to write down a likelihood function for the observed image
  given the source and image model parameters, and \thetractor\ can
  optimize or sample from this likelihood.  This allows us to frame
  the measurement of astronomical sources (in multiple images, perhaps
  taken by multiple instruments) as an optimization problem, as
  opposed to the traditional approach of using \emph{estimators}.
  \thetractor\ software allows parameters to be held fixed during the
  fitting, enabling a variety of different fitting procedures,
  including \emph{forced photometry}, where the source profiles are
  held fixed and only the brightness in a new image is measured.
\end{abstract}

%% Keywords should appear after the \end{abstract} command. 
%% The AAS Journals now uses Unified Astronomy Thesaurus concepts:
%% https://astrothesaurus.org
%% You will be asked to selected these concepts during the submission process
%% but this old "keyword" functionality is maintained in case authors want
%% to include these concepts in their preprints.
\keywords{Classical Novae (251) --- Ultraviolet astronomy(1736)}

\section{Introduction} \label{sec:intro}

- catalog matching is bad

- coadds are bad

- generative models -- extract all information

- masking no problem

- working in image space

\cite{2009AJ....137.4400L}

\section{Methods}

%\subsection{Components}

The two main elements of \thetractor\ are an \emph{image} and a
\emph{source}.  An image includes the pixel data, the noise model, and
calibration information such as the point-spread model, background
model, and photometric and astrometric calibrations.  A source
includes a position and brightness, and additional parameters
depending on the type of source.  For example, a star might include
motion or variability, while a galaxy model would include a spatial
profile.

%\subsection{Rendering}

The core functionality of \thetractor\ is \emph{rendering} a
pixel-by-pixel predicted or \emph{model} image patch for a source in
an image.  This requires transforming from the source parameters into
image units (pixels and counts); these transformations are performed
by \emph{calibration} objects that describe the properties of an
image.

For example, we must convert the position of a source into a pixel
location in the model image.  If the source position is represented as
(RA,Dec) coordinates, then the corresponding image calibration object
is a World Coordinate System object that converts celestial
coordinates into pixel coordinates.  Similarly, the brightness of a
source can be represented in magnitudes or linear flux units, and then
the image's photometric calibration object applies a zeropoint to
arrive at the count units of the image.  Other calibration objects
include the point-spread function model and the sky background model.

\thetractor\ is largely agnostic about exactly which parameters and
calibrations are used for the position and brightness parameters; as
long as the source and image calibrations are consistent, different
parameterizations can be chosen to suit the problem.


Rendering a source into an image then involves the steps of computing
the center of the source in pixel coordinates, finding the PSF for
that location in the image (perhaps depending on properties of the
source, for example if a spectrally-varying PSF model is available),
and the total number of counts contributed by the source.  The source
object is responsible for rendering a unit-flux




\subsection{Implementation details}

- Gaussian mixture models; Sersic models

- FFT galaxy convolution

- Optimizers

\section{Example results}

% \subsection{Refitting SDSS???} no not compelling


\subsection{Legacy Surveys}

\subsection{legacyhalos / WISE / GALEX forced photometry}

- time-resolved WISE forced phot

\subsection{Sampling from joint distribution of blended galaxy shapes???}

\subsection{Image subtraction / transient detection / fitting moving objects}

\subsection{Can we measure proper motions below the Gaia limit?}



\section{Discussion}

- what is and isn't in The Tractor (source detection; model selection; PSF models; flats, darks, weights, masking; blobs)

- extensions (Poisson noise model?  Tiered coadds?)

- will it replace Source Extractor and why that is not our goal (aka why we hate success)


%\begin{acknowledgments}
%\end{acknowledgments}

\software{%Source Extractor \citep{1996A&AS..117..393B}
          }

%\appendix
%\section{Appendix information}

\bibliography{tractor}{}
\bibliographystyle{aasjournal}

\end{document}
