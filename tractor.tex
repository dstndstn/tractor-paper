\documentclass[linenumbers]{aastex631}

\shorttitle{The Tractor}
\shortauthors{Lang and Hogg}

\newcommand{\thetractor}{\emph{The Tractor}}

\begin{document}

\title{\emph{The Tractor}: forward modeling of astronomical images}

\author[0000-0002-1172-0754]{Dustin A. Lang}
\affiliation{Perimeter Institute for Theoretical Physics, 31 Caroline Street N, Waterloo, ON N25 2YL, Canada}

\author[0000-0003-2866-9403]{David W. Hogg}
\affiliation{Center for Computational Astrophysics, Flatiron Institute, 162 Fifth Avenue, New York, NY 10010, USA}
\affiliation{Max-Planck-Institut f\"ur Astronomie, K\"onigstuhl 17, D-69117 Heidelberg, Germany}
\affiliation{Center for Cosmology and Particle Physics, Department of Physics, New York University, 726 Broadway, New York, NY 10003, USA}

\begin{abstract}
  We present \thetractor, a software package for performing forward or
  \emph{generative} modeling in astronomical images.  Given a set of
  astronomical sources and a model of the properties of an
  astronomical image, \thetractor\ generates a ``model image'': a
  prediction of what would be observed in the image.  In the common
  case where a noise model for the image is available (for example,
  typical ground-based background-limited optical images), it is
  simple to write down a likelihood function for the observed image
  given the source and image model parameters, and \thetractor\ can
  optimize or sample from this likelihood.  This allows us to frame
  the measurement of astronomical sources (in multiple images, perhaps
  taken by multiple instruments) as an optimization problem, as
  opposed to the traditional approach of using \emph{estimators}.
  \thetractor\ software allows parameters to be held fixed during the
  fitting, enabling a variety of different fitting procedures,
  including \emph{forced photometry}, where the source profiles are
  held fixed and only the brightness in a new image is measured.
\end{abstract}

%% Keywords should appear after the \end{abstract} command. 
%% The AAS Journals now uses Unified Astronomy Thesaurus concepts:
%% https://astrothesaurus.org
%% You will be asked to selected these concepts during the submission process
%% but this old "keyword" functionality is maintained in case authors want
%% to include these concepts in their preprints.
\keywords{Classical Novae (251) --- Ultraviolet astronomy(1736)}

\section{Introduction} \label{sec:intro}

- generative models

\cite{2009AJ....137.4400L}

\section{Software structure}

\section{Implementation details}

- FFT galaxy convolution

- Sersic models

- Optimizers

\section{Example results}

\subsection{Refitting SDSS???}

\subsection{Legacy Surveys}

\subsection{legacyhalos / WISE / GALEX forced photometry}

\subsection{Sampling from joint distribution of blended galaxy shapes???}


\section{Discussion}

- what is and isn't in The Tractor (source detection; model selection; PSF models; flats, darks, weights, masking)

- extensions (Poisson noise model?)

- will it replace Source Extractor and why that is not our goal (aka why we hate success)


%\begin{acknowledgments}
%\end{acknowledgments}

\software{%Source Extractor \citep{1996A&AS..117..393B}
          }

%\appendix
%\section{Appendix information}

\bibliography{tractor}{}
\bibliographystyle{aasjournal}

\end{document}
